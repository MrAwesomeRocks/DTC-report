\chapter{Introduction}
\label{chap:intro}


Patients who have suffered from a stroke or traumatic brain injuries can develop a condition called left side neglect. This condition is caused by the inability of the right side of the brain to process information that comes from the left side of the body. Patients therefore struggle to engage with the left side of their body. Our project specifically addresses the patients who struggle to continue turning their head past the midline of the body. Our project partner, physical therapist Dr. Kate Enzler, stated during the client interview that the typical patient often forgets there is a whole world to the left side of the body. This lack of head rotation can cause everyday challenges for the patient as they may bump into things on the left side of them, miss information when reading, and many other problems. The solution to this problem is preferably wearable as per our client’s request. Since left neglect patients span a wide demographic, the solution must be customizable to fit differently sized patients.

The research phase of this project consisted of both secondary and primary research. While researching the condition of left neglect, we also researched existing products for the condition and their shortcomings in order to avoid those elements in our design. Current methods of treatment include neck vibration therapy, prism therapy, eye patching, optokinetic stimulation, and the use of tangible stimuli to remind a patient to visually scan left. Many of these visual stimuli were effective in the moment but did not have long term effects. Additionally, many of the products, despite being effective, required a lot of work with the therapist in order to understand how the device worked or what its purpose was. Many of these solutions also place a heavy reliance of the patient onto their caregiver. Therefore, the patients are heavily reliant on others, and lose their autonomy. The goal of the product that we designed was to aid patients in regaining their autonomy while continuing to receive cues, in a haptic form, to visually scan left.  Our solution is a haptic motor clip that attaches on a patient's hat or shirt to a point of skin contact on the right side of their body, which serves as a haptic stimulus to remind the patient to visually scan left. The motor will connect via bluetooth to an app, which will control the motor. The final part of the solution will consist of a pair of glasses that have cue lines on the left side that guide the patients towards the left. The glasses also have a gyroscope on them that track motion and will communicate with the app, and turn the motor off when the patient has turned their head to a certain degree. 

%%% Local Variables:
%%% mode: latex
%%% TeX-master: "../final_report"
%%% End:
