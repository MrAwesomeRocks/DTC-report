\chapter{Interviews Summary}
\label{chap:interviews}

On Thursday, January 12th 2023, we had our initial client interview with
Dr. Kate Enzler, a physical therapist at the Ascension Alexian Brothers
Rehabilitation Center. The interview was conducted via Zoom at 8:15 am in the
Ford Building. The purpose of this interview was to learn more about left side
neglect and its effects on patients. It was also used to clarify what the goal
of our project should be and how current solutions are not sufficient. This
appendix summarizes what we learned about the design problem, requirements,
users, and current solutions.

\section{Problems}

Our client explained the current challenges of patients with left neglect and
clarified what type of patients we would be working with.
\begin{itemize}
\item Patients with left neglect have varying awareness of the world on the
  left side of their body. However, they do have all visual, auditory and
  physical abilities on the left side.
\item The patients we will be working with have the ability to scan to the
  midline of their body.
\item The patients often collide with objects or people on the left side of
  their body which is very dangerous as it can injure them. 
\item The current treatments are very dependent on caregivers or therapists as
  it requires verbal cues from the caregiver and for the caregiver to be
  present at all times. One goal of our solutions should be to create more
  independence for the patient after being taught to use the design.
\end{itemize}

\section{Requirements}

Our client identified these requirements for the design:
\begin{description}
\item[Adjustable] The design should be able to change ``sizes'' in order to be
  able to accommodate many different patients' body sizes.  It will also have
  different time intervals that can be adjusted so that each patient can have
  stimulus at a rate that is set for their specific needs.
\item[Affordable] The design should cost less than \$100, making it affordable
  for all users. This is necessary so that the product can be used by all
  demographics of patients.
\item[Lightweight] The product must be lightweight so that all users, no matter
  their strength/phase of their recovery, should be able to use it.
\item[Discrete] The design should not limit the user in any way and should not
  make the user stand out in public. This way the user is comfortable with
  using the product and the product preserves their dignity.
\item[Easy-to-Use] Users of all ages should be able to use/navigate the
  device. The device should not contain technology. This way the user does not
  need to receive aid in order to use the product like they do using current
  therapies and devices (creating autonomy for the user).
\end{description}

\section{Users}

We also gathered the following general information about our users:
\begin{itemize}
\item Users of the product include any patients with left neglect past the
  midline of the body.
\item Users are typically elderly or middle aged and might be overweight.
\item Users stay at the rehabilitation center for an average of 3 weeks.
\end{itemize}

\section{Current Solutions}

Dr. Enzler described the current solutions available to help prompt left side
awareness for patients with left neglect. 
\begin{description}
\item[Visual Scanning] Having a red line or ruler on the left side of the body
  or an object (e.g. book) in order to catch the patient's attention (prompted
  to find the red line). Dr. Enzler commented that this was relatively
  effective for specific activities (like reading a book) however not
  universally able to be used across multiple activities.
\item[Eye Patching] Placing an eye patch over the patient’s right eye to
  encourage left side awareness (hit or miss with patients).
\item[Prism Glasses] Glasses that bring objects in the left field of vision to
  the midline. 
\item[``Lighthouse Technique''] The patient is prompted by the
  caregiver/therapist to scan the surroundings like a lighthouse. Dr. Enzler
  uses this method most and described it as the most effective.
\end{description}

\section{Conclusion}

The interview provided critical information for understanding the problem,
users, and client requirements. When we do our user observation at Shirley Ryan
Rehabilitation Center, we will continue to learn more about left side neglect
and ideate potential solutions.

%%% Local Variables:
%%% mode: latex
%%% TeX-master: "../final_report"
%%% End:

